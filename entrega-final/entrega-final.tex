\documentclass[
  article,
  a4paper,
  12pt,
  oneside,
  fleqn
]{abntex2}

% Pacote de formatação personalizado que aparece na sua imagem
\usepackage[ABNTNum=brkt]{entrega-final}

% Arquivo de bibliografia
\addbibresource{entrega-final.bib}

\titulo{PROPOSTA DE UM SISTEMA INTELIGENTE PARA BUSCA E COMPARAÇÃO DE PREÇOS DE PRODUTOS DA CESTA BÁSICA}

\autor{
  Alexandre Reitemeyer\thanks{\affil{Curso de Sistemas de Informação, UNOESC, Chapecó}\sep\email{alexandre.reitemeyer@unoesc.edu.br}} \\
  Angelo Pandolfo Antonini\thanks{\affil{Curso de Sistemas de Informação, UNOESC, Chapecó}\sep\email{aluno2@edu.unoesc.br}} \\
  Jessé Chaves\thanks{\affil{Curso de Sistemas de Informação, UNOESC, Chapecó}\sep\email{aluno3@edu.unoesc.br}} \\
  Lucas Santos Magro\thanks{\affil{Curso de Sistemas de Informação, UNOESC, Chapecó}\sep\email{lucas.magro@unoesc.edu.br}} \\
  Prof. Jacson Luiz Matte\thanks{\affil{Orientador; Curso de Sistemas de Informação, UNOESC, Chapecó}\sep\email{jacson.matte@unoesc.edu.br}}
}

\data{Chapecó, 05 de dezembro de 2025}

\begin{document}

\nocite{*}

\pretextual

\begin{paginadetitulo}
    \begin{ambienteresumo}
        Este trabalho propõe o desenvolvimento de uma solução computacional composta por um aplicativo e um site que visa simplificar e automatizar a busca por produtos da cesta básica. Utilizando recursos de inteligência artificial, o sistema permitirá ao usuário pesquisar ou escanear produtos de seu interesse, recebendo uma análise comparativa de preços e estabelecimentos. A plataforma classificará as opções com base em múltiplos critérios, como preço, distância e avaliações de outros usuários, oferecendo uma ferramenta poderosa para a tomada de decisão, promovendo economia de tempo e dinheiro.
        \palavraschave{Cesta Básica. Comparação de Preços. Inteligência Artificial. Aplicativo Móvel.}
    \end{ambienteresumo}
\end{paginadetitulo}

\textual
\newpage

\section{Introdução}

O custo de vida é uma preocupação constante para as famílias brasileiras, sendo a aquisição de itens da cesta básica uma das despesas mais significativas no orçamento mensal. Em um cenário econômico de flutuação de preços e ampla variedade de estabelecimentos comerciais, tanto físicos quanto online, a tarefa de encontrar os melhores preços para produtos essenciais pode ser complexa e demorada. Os consumidores frequentemente gastam um tempo considerável pesquisando em diferentes supermercados, comparando ofertas e calculando a melhor opção de compra para maximizar sua economia.

Com o avanço da tecnologia e a crescente digitalização dos serviços, surgem oportunidades para otimizar processos cotidianos. O uso de aplicativos móveis e plataformas web para compras tornou-se comum, porém, poucas soluções integram de forma eficiente a busca e comparação de preços de produtos essenciais em um único ambiente, especialmente com foco regional e com o auxílio de tecnologias emergentes.

Nesse contexto, este trabalho propõe o desenvolvimento de uma solução computacional composta por um aplicativo e um site que visa simplificar e automatizar a busca por produtos da cesta básica. Utilizando recursos de inteligência artificial, o sistema permitirá ao usuário pesquisar ou escanear produtos de seu interesse, recebendo uma análise comparativa de preços e estabelecimentos. A plataforma classificará as opções com base em múltiplos critérios, como preço, distância, acessibilidade e avaliações de outros usuários, oferecendo uma ferramenta poderosa para a tomada de decisão, promovendo economia de tempo e dinheiro.

\section{Delimitação do Tema e Justificativa}

\subsection{Delimitação do Tema}

O presente estudo concentra-se no desenvolvimento de um sistema de software, abrangendo uma plataforma web e um aplicativo móvel, destinado à busca e comparação de preços de produtos que compõem a cesta básica brasileira.

O escopo do projeto inclui:
\begin{itemize}
    \item A criação de uma interface que permita ao usuário buscar produtos por nome ou através do escaneamento de seu código de barras.
    \item A implementação de um algoritmo de inteligência artificial para coletar, processar e classificar dados de preços de diferentes fontes, incluindo estabelecimentos comerciais locais e plataformas de e-commerce.
    \item A apresentação dos resultados ao usuário de forma organizada, com base em critérios personalizáveis como menor preço, menor distância, melhor avaliação e acessibilidade do estabelecimento.
\end{itemize}

O projeto não abrangerá o desenvolvimento do sistema de e-commerce em si, ou seja, não incluirá funcionalidades de pagamento, logística de entrega ou gerenciamento de estoque dos estabelecimentos. O foco é estritamente na disponibilização de informações para auxiliar o consumidor em sua decisão de compra.

\subsection{Justificativa}

A relevância deste projeto pode ser analisada sob três perspectivas principais: econômica, social e tecnológica.

Do ponto de vista econômico, a plataforma oferece um benefício direto ao consumidor. Em um país onde a inflação dos alimentos impacta significativamente o poder de compra, uma ferramenta que facilita a economia em itens essenciais tem grande valor prático. Ao otimizar o processo de compra, o sistema contribui para uma melhor gestão do orçamento familiar.

Na esfera social, o projeto promove o consumo consciente e a democratização do acesso à informação. Ele capacita o consumidor a tomar decisões mais informadas, fomentando a competitividade entre os estabelecimentos comerciais e, potencialmente, levando a uma maior transparência de preços no mercado local. Além disso, a inclusão de critérios como acessibilidade nos estabelecimentos agrega um valor social importante.

Sob o aspecto tecnológico, o trabalho se destaca pela aplicação de inteligência artificial em um problema cotidiano e de grande alcance. A integração de busca inteligente, processamento de dados em tempo real e geolocalização em uma plataforma unificada representa uma inovação no setor de aplicativos de comparação de preços. O desenvolvimento desta solução contribui para a exploração de novas tecnologias na resolução de desafios práticos da sociedade.

\section{Objetivo Geral}

Desenvolver uma plataforma computacional, composta por um aplicativo móvel e um website, que utilize inteligência artificial para auxiliar os usuários na busca e comparação de preços de produtos da cesta básica, classificando estabelecimentos físicos e online com base em critérios como preço, distância e avaliações, a fim de otimizar a economia e o tempo do consumidor no processo de compras.

\section{Fundamentação Teórica}

Os trabalhos analisados demonstram excelência na engenharia de software aplicada, selecionando stacks tecnológicas robustas para resolver problemas distintos. O projeto de gestão de doações ("Cesta Beneficente") adota uma arquitetura moderna baseada em Angular (TypeScript) no front-end e Node.js com Express e PostgreSQL no back-end, garantindo escalabilidade. Em contrapartida, o aplicativo de monitoramento econômico ("Observatório Econômico") aposta no desenvolvimento nativo Android (Java) integrado ao Firebase, uma escolha estratégica para garantir a persistência de dados e sincronização em nuvem.

No âmbito da logística, o sistema de doações se destaca ao implementar uma solução para o Problema do Caixeiro Viajante (TSP). Utilizando a API Directions do Google Maps, o software não apenas traça rotas, mas otimiza a ordem dos pontos de parada (waypoints). Isso permite que o sistema calcule e exiba o tempo e a distância total das entregas com precisão, oferecendo ao voluntário a opção de gerar rotas inteligentes que economizam recursos.

Já o aplicativo de coleta de preços foca na integridade e tempestividade da informação. Ele substitui o método manual (papel e Excel), propenso a erros de transcrição, por um sistema de coleta digital que valida os campos obrigatórios instantaneamente. Um diferencial crucial é o suporte ao funcionamento offline, onde os dados são armazenados localmente e sincronizados automaticamente com o banco NoSQL Firestore assim que a conexão é restabelecida, essencial para pesquisas em regiões com internet instável.

Ambos os projetos realizaram levantamentos rigorosos de Requisitos Funcionais e Não Funcionais. O sistema de doações detalha desde o bloqueio de auto-benefício por administradores até a geração de relatórios de entregas. O aplicativo econômico foca na exportação de dados para planilhas compatíveis com Excel para análise posterior. A segurança também é prioritária: o primeiro utiliza criptografia JWT para autenticação, enquanto o segundo emprega o Firebase Authentication para controle de acesso de pesquisadores. Além disso, ambos investiram na Experiência do Usuário (UX) com feedbacks visuais (como "Toasts" de erro ou sucesso) para melhorar a comunicação com o operador.

Por fim, a referência à página governamental é citada como estrutural para a metodologia científica. O artigo sobre o monitoramento de preços baseia-se explicitamente no Decreto-Lei nº 399 de 1938, que define as quantidades e os 13 itens da "ração essencial mínima" por região (como a Região 3 para o Mato Grosso). Essa base legal padroniza a coleta, permitindo que os dados obtidos pelo aplicativo sejam comparáveis e consistentes com metodologias oficiais como a do DIEESE.

\section{Metodologia e Proposta do Sistema}

\subsection{Requisitos Funcionais}

\begin{itemize}
    \item Permitir o login de usuários e entidades da comunidade.
    \item Permitir o cadastro de usuários e entidades da comunidade.
    \item Permitir o cadastro de produtos, serviços e eventos locais.
    \item Permitir busca sobre produtos, serviços e eventos locais.
    \item Permitir que o usuário escaneie o produto para busca.
    \item Exibir informações sobre produtos, serviços e eventos locais.
    \item Enviar notificações ou mensagens entre os participantes.
    \item Exibir métricas dos produtos, serviços e eventos locais.
    \item Permitir análise comparativa de preços.
    \item Permitir que o usuário crie listas de compras.
    \item Deve integrar inteligência artificial para otimizar processo de escaneamento.
    \item Deve garantir a exclusão dos dados de usúario e entidades que deletarem suas contas.
\end{itemize}

\subsection{Requisitos Não Funcionais}

\begin{itemize}
    \item Deve apresentar interface simples e de fácil navegação.
    \item Deve apresentar compatibilidade com dispositivos móveis.
    \item Deve garantir tempo de resposta rápido.
    \item Deve proteger os dados dos usuários.
    \item Deve ser escalável para novas melhorias.
    \item É desejável que o sistema permita a criação de campanhas de arrecadação.
\end{itemize}

\subsection{Modelagem do Sistema}

A seguir são apresentados os diagramas da Unified Modeling Language (UML) que detalham a estrutura e o comportamento do sistema proposto.

\begin{figure}[htb]
    \caption{\label{fig:classes}Diagrama de Classes}
    \begin{center}
        \includegraphics[width=16cm]{UMLs/diagrama_de_classe.png}
    \end{center}
    \fonte{Acervo Pessoal (2025).}
\end{figure}

\begin{figure}[htb]
    \caption{\label{fig:caso_uso}Diagrama de Casos de Uso}
    \begin{center}
        \includegraphics[width=16cm]{UMLs/caso_de_uso.png}
    \end{center}
    \fonte{Acervo Pessoal (2025).}
\end{figure}

\begin{figure}[htb]
    \caption{\label{fig:seq_lista}Diagrama de Sequência: Criar Lista}
    \begin{center}
        \includegraphics[width=16cm]{UMLs/sequencia_criar_lista.png}
    \end{center}
    \fonte{Acervo Pessoal (2025).}
\end{figure}

\begin{figure}[htb]
    \caption{\label{fig:seq_escanear}Diagrama de Sequência: Escanear Produto}
    \begin{center}
        \includegraphics[width=16cm]{UMLs/sequencia_escanear_produto.png}
    \end{center}
    \fonte{Acervo Pessoal (2025).}
\end{figure}

\begin{figure}[htb]
    \caption{\label{fig:atividade}Diagrama de Atividades}
    \begin{center}
        \includegraphics[width=16cm]{UMLs/diagrama_de_atividade.png}
    \end{center}
    \fonte{Acervo Pessoal (2025).}
\end{figure}

\begin{figure}[htb]
    \caption{\label{fig:componente}Diagrama de Componente}
    \begin{center}
        \includegraphics[width=16cm]{UMLs/diagrama_componente.png}
    \end{center}
    \fonte{Acervo Pessoal (2025).}
\end{figure}

\clearpage

\section{Conclusão}

O presente trabalho apresentou a proposta de desenvolvimento de um sistema inteligente para busca e comparação de preços de produtos da cesta básica, visando mitigar as dificuldades enfrentadas pelos consumidores diante da oscilação econômica e da dispersão de ofertas no mercado. A solução concebida, composta por um aplicativo móvel e uma interface web, utiliza tecnologias de inteligência artificial para modernizar o processo de compra, tornando-o mais eficiente e econômico.

Através do levantamento de requisitos funcionais e não funcionais, bem como da modelagem do sistema utilizando a linguagem UML (Diagramas de Casos de Uso, Classes, Sequência, Atividades e Componentes), foi possível estruturar logicamente a solução. A modelagem demonstrou a viabilidade técnica do projeto, clarificando os fluxos de interação do usuário — desde o escaneamento do produto até a geração de listas comparativas — e a arquitetura de dados necessária para suportar tais operações.

Conclui-se que a implementação deste sistema possui grande relevância social e econômica. Ao centralizar informações de preços, distância e avaliações, a ferramenta não apenas empodera o consumidor na tomada de decisão, mas também fomenta a competitividade transparente entre os estabelecimentos comerciais. Como trabalhos futuros, sugere-se o desenvolvimento do protótipo funcional com base na modelagem apresentada, seguido pela realização de testes de usabilidade e a integração gradual dos algoritmos de inteligência artificial propostos.

\postextual
\newpage
\printbibliography

\end{document}