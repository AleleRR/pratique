\documentclass[
  article,
  a4paper,
  12pt,
  oneside,
  fleqn
]{abntex2}

\usepackage[ABNTNum=brkt]{unoesc-article}

\addbibresource{unoesc-article.bib}

\titulo{PROPOSTA DE UM SISTEMA INTELIGENTE PARA BUSCA E COMPARAÇÃO DE PREÇOS DE PRODUTOS DA CESTA BÁSICA}

\autor{
  Alexandre Reitemeyer\thanks{\affil{Curso de Sistemas de Informação, UNOESC, Chapecó}\sep\email{alexandre.reitemeyer@unoesc.edu.br}} \\
  Angelo Pandolfo Antonini\thanks{\affil{Curso de Sistemas de Informação, UNOESC, Chapecó}\sep\email{aluno2@edu.unoesc.br}} \\
  Jessé Chaves\thanks{\affil{Curso de Sistemas de Informação, UNOESC, Chapecó}\sep\email{aluno3@edu.unoesc.br}} \\
  Lucas Santos Magro\thanks{\affil{Curso de Sistemas de Informação, UNOESC, Chapecó}\sep\email{lucas.magro@unoesc.edu.br}} \\
  Prof. Jacson Luiz Matte\thanks{\affil{Orientador; Curso de Sistemas de Informação, UNOESC, Chapecó}\sep\email{jacson.matte@unoesc.edu.br}}
}

\data{Chapecó, 26 de setembro de 2025}

\begin{document}

\pretextual

\begin{paginadetitulo}
    \begin{ambienteresumo}
        Este trabalho propõe o desenvolvimento de uma solução computacional composta por um aplicativo e um site que visa simplificar e automatizar a busca por produtos da cesta básica. Utilizando recursos de inteligência artificial, o sistema permitirá ao usuário pesquisar ou escanear produtos de seu interesse, recebendo uma análise comparativa de preços e estabelecimentos. A plataforma classificará as opções com base em múltiplos critérios, como preço, distância e avaliações de outros usuários, oferecendo uma ferramenta poderosa para a tomada de decisão, promovendo economia de tempo e dinheiro.
        \palavraschave{Cesta Básica. Comparação de Preços. Inteligência Artificial. Aplicativo Móvel.}
    \end{ambienteresumo}
\end{paginadetitulo}

\textual
\newpage

\section{Introdução}

O custo de vida é uma preocupação constante para as famílias brasileiras, sendo a aquisição de itens da cesta básica uma das despesas mais significativas no orçamento mensal. Em um cenário econômico de flutuação de preços e ampla variedade de estabelecimentos comerciais, tanto físicos quanto online, a tarefa de encontrar os melhores preços para produtos essenciais pode ser complexa e demorada. Os consumidores frequentemente gastam um tempo considerável pesquisando em diferentes supermercados, comparando ofertas e calculando a melhor opção de compra para maximizar sua economia.

Com o avanço da tecnologia e a crescente digitalização dos serviços, surgem oportunidades para otimizar processos cotidianos. O uso de aplicativos móveis e plataformas web para compras tornou-se comum, porém, poucas soluções integram de forma eficiente a busca e comparação de preços de produtos essenciais em um único ambiente, especialmente com foco regional e com o auxílio de tecnologias emergentes.

Nesse contexto, este trabalho propõe o desenvolvimento de uma solução computacional composta por um aplicativo e um site que visa simplificar e automatizar a busca por produtos da cesta básica. Utilizando recursos de inteligência artificial, o sistema permitirá ao usuário pesquisar ou escanear produtos de seu interesse, recebendo uma análise comparativa de preços e estabelecimentos. A plataforma classificará as opções com base em múltiplos critérios, como preço, distância, acessibilidade e avaliações de outros usuários, oferecendo uma ferramenta poderosa para a tomada de decisão, promovendo economia de tempo e dinheiro.

\section{Delimitação do Tema e Justificativa}

\subsection{Delimitação do Tema}

O presente estudo concentra-se no desenvolvimento de um sistema de software, abrangendo uma plataforma web e um aplicativo móvel, destinado à busca e comparação de preços de produtos que compõem a cesta básica brasileira.

O escopo do projeto inclui:
\begin{itemize}
    \item A criação de uma interface que permita ao usuário buscar produtos por nome ou através do escaneamento de seu código de barras.
    \item A implementação de um algoritmo de inteligência artificial para coletar, processar e classificar dados de preços de diferentes fontes, incluindo estabelecimentos comerciais locais e plataformas de e-commerce.
    \item A apresentação dos resultados ao usuário de forma organizada, com base em critérios personalizáveis como menor preço, menor distância, melhor avaliação e acessibilidade do estabelecimento.
\end{itemize}

O projeto não abrangerá o desenvolvimento do sistema de e-commerce em si, ou seja, não incluirá funcionalidades de pagamento, logística de entrega ou gerenciamento de estoque dos estabelecimentos. O foco é estritamente na disponibilização de informações para auxiliar o consumidor em sua decisão de compra.

\subsection{Justificativa}

A relevância deste projeto pode ser analisada sob três perspectivas principais: econômica, social e tecnológica.

Do ponto de vista econômico, a plataforma oferece um benefício direto ao consumidor. Em um país onde a inflação dos alimentos impacta significativamente o poder de compra, uma ferramenta que facilita a economia em itens essenciais tem grande valor prático. Ao otimizar o processo de compra, o sistema contribui para uma melhor gestão do orçamento familiar.

Na esfera social, o projeto promove o consumo consciente e a democratização do acesso à informação. Ele capacita o consumidor a tomar decisões mais informadas, fomentando a competitividade entre os estabelecimentos comerciais e, potencialmente, levando a uma maior transparência de preços no mercado local. Além disso, a inclusão de critérios como acessibilidade nos estabelecimentos agrega um valor social importante.

Sob o aspecto tecnológico, o trabalho se destaca pela aplicação de inteligência artificial em um problema cotidiano e de grande alcance. A integração de busca inteligente, processamento de dados em tempo real e geolocalização em uma plataforma unificada representa uma inovação no setor de aplicativos de comparação de preços. O desenvolvimento desta solução contribui para a exploração de novas tecnologias na resolução de desafios práticos da sociedade.

\section{Objetivo Geral}

Desenvolver uma plataforma computacional, composta por um aplicativo móvel e um website, que utilize inteligência artificial para auxiliar os usuários na busca e comparação de preços de produtos da cesta básica, classificando estabelecimentos físicos e online com base em critérios como preço, distância e avaliações, a fim de otimizar a economia e o tempo do consumidor no processo de compras.

\postextual
\printbibliography

\end{document}